{\textit{Бахтин М. М.} Проблема текста в лингвистике, филологии и других гуманитарных науках. Опыт философского анализа// Эстетика словесного творчества. М., 1986. С. 301}
{\textit{Ильин И. П.} Постмодернизм от истоков до конца столетия: эволюция научного мифа. М., 1998. С. 166}
{\textit{Давлетьярова А. Т.} Особенности синтаксиса предложения в прозе русского постмодернизма: на материале произведений С. Соколова, Т. Толстой и В. Сорокина: автореф. дис. ... канд. фил. наук. СПб, 2007}
{Термин Р. Барта: \textit{Барт Р.} Смерть автора// Избранные работы. Семиотика. Поэтика. М., 1989. С. 387~---388}
{\textit{Богданова О. В.} Постмодернизм: к истории явления и его органичности современному русскому литературному процессу// Вестник Санкт-Петербургского университета. Серия 2. Выпуск 2. СПб, 2003. С. 84 [цит. по: \textit{Давлетьярова} 2007]}
{\textit{Зорина Е. С.} К вопросу о синтаксической структуре современного художественного текста. Вестник СПбГУ. Сер. 9. Вып. 3. СПб., 2013. С.160}
{\textit{Зорина Е. С.} Указ. соч. С.165}
{\textit{Николина Н. А.} Активные процессы в языке современной художественной литературы. М., 2009. С. 7}
{\textit{Николина Н. А.}Активные процессы в языке современной художественной литературы. М., 2009. С. 238}
{\textit{Арутюнова Н. Д.} О синтаксических типах художественной прозы// Синтаксис современного русского языка. Хрестоматия с заданиями. Спб, 2013. С. 681}
{\textit{Акимова Г. Н.} Новое в синтаксисе современного русского языка. М., 1990. С. 160}
{\textit{Мартьянова И. А. }Функционирование синтаксических единиц в современной русской прозе// Вестник Череповецкого государственного университета. 2012. № 2 (37). С. 82}
{\textit{Руднев В. П.}Словарь культуры XX века. М.,1997. С.238~---240}
{\textit{Звягина М. Ю. }{Жанровые трансформации в русской прозе второй половины XX~--- нач. XXI в. Астрахань, 2016. С.28}}
{{ }\textit{Маркова Т. Н.} Авторские жанровые номинации в современной русской прозе как показатель кризиса жанрового сознания// Вопросы литературы. 2001. №1.}
{там же}
{\textit{Тюпа В. И.} Дискурс/Жанр // Глава1. Жанр и дискурс. М., 2013. См.: С. 16~---17}
{\textit{Звягина М. Ю.} Жанровые трансформации в русской прозе второй половины XX—нач. XXI века. Астрахань, 2016. С. 43}
{там же}
{\textit{Синенко В. С.} Особенности жанровой системы современной советской литерату-ры. Уфа, 1986.  С. 59.}
{там же}
{{Российский энциклопедический словарь/ Под ред. \textit{А. М. Прохорова}. М., 2001 }\par
    {Ссылка на электронный ресурс:
    https://clck.ru/AuxNo}}
{{Большая советская энциклопедия/ cловар. cтатья: \textit{Соколов Н. Н., Скородумова Н. П.} М., 3-е изд. 1970~--- 1978}\par{Ссылка на электронный ресурс: https://slovar.cc/enc/bse/2044790.html}}
{\textit{{Нечаева И. В., Захаренко Е. Н., Комарова Л. Н. }}{Новый словарь иностранных слов. М., 2008. 

    Ссылка на электронный ресурс: https://slovar.cc/rus/inostr-nov/1431100.html}}
{\textit{Бочоришвили Е.} Только ждать и смотреть// М., 2015}
{\textit{Иванин М. И. }О стенографии или искусстве скорописи и применении её к русскому языку. СПб., 1858}
{{ Цит. по: }\textit{{Юрковский А. М.
      }}{Стенография сквозь века. М., 1969. С.55}}
{\textit{{Виноградов В. В. }}{«Стиль «Пиковой дамы» // Временник Пушкинской комиссии, т. 2, 1936. С.75—86.}}
{{Термин «языковая личность» используется в значении: «Личность, выраженная в языке (текстах) и через язык; личность, реконструированная в основных своих чертах на базе языковых средств». }\textit{{Караулов Ю.Н. }}{Русский язык и языковая личность. М., 1987. С.38}}
{\textit{{Падучева Е. В. }}{Семантические исследования. Семантика времени и вида в русском языке. Семантика нарратива.~--- М.: Языки славянской культуры, 1996. Изд. 2-е, 2010. С.201—202}}
{\textit{{Заика В. И. }}{Повествователь как компонент художественной модели //Говорящий и слушающий: Языковая личность, текст, проблемы обучения. СПб, 2001. C.384}}
{\textit{{Барт Р.}}{ Избранные работы: Семиотика. Поэтика. М., 1994. С.387}}
{\textit{{Гончарова Е. А., Шишкина И. П. }}{Интерпретация текста. Немецкий язык. М., Высшая школа, 2005. 368 с. }}
{\textit{{Валгина Н. С. }}{Авторская модальность. Образ автора // Теория текста. М., 2004. С.96—97}}
{\textit{{Гальперин И. Р. }}{Текст как объект лингвистического исследования. М, 2006. С.117}}
{\textit{{Валгина Н. С. }}{Авторская модальность. Образ автора // Теория текста. М., 2004. С.96}}
{\textit{Золотова Г. А.}О традициях и тенденциях в современной грамматической науке//Синтаксис современного русского языка. Хрестоматия с заданиями. СПб, 2013. С.31}
{\textit{Падучева Е. В. }В. В. Виноградов и наука о языке художественной прозы // Известия РАН. Серия литературы и языка. Т. 54. 1995. №3. С.43}
{\textit{Падучева Е. В. }Семантические исследования. С.200—201}
{\textit{ Падучева Е. В. }Семантические исследования. С.201}
{\textit{ Падучева Е. В.} В. В. Виноградов и наука о языке художественной прозы. См.: С.44}
{\textit{Падучева Е. В.}Семантические исследования. См.: С.298—299}
{\textit{ Падучева Е. В. }Семантические исследования. С.297}
{\textit{Падучева Е. В.}Семантические исследования. С.299}
{\textit{Падучева Е. В.} Семантические исследования. С.262—265}
{\textit{Падучева Е. В.}Семантические исследования. С.203—204}
{\textit{ Шведова Н. Ю., Лопатина В. В.} Русская грамматика. М., 1990. С.499}
{\textit{Падучева Е. В.} Семантические исследования. См.: С.299—300}
{\textit{Падучева Е. В. }Семантические исследования. См.: С.300}
{\textit{Шведова Н. Ю., Лопатина В. В.}{Русская грамматика. М., 1990. С.504}}
{\textit{Акимова Г. Н.} Новое в синтаксисе современного русского языка. М., 1990. С.43}
{\textit{Арутюнова Н. Д.}О синтаксических типах художественной прозы. Общее языкознание. М., 1972.}
{\textit{{ Акимова Г. Н. }}{Размер предложения как фактор стилистики и грамматики// Вопросы языкознания. 1973. №2. С.79}}
{\textit{{ Адмони В. Г.}}{ Размер предложения и словосочетания как явление синтаксического строя// Вопросы языкознания. 1966. №4. С.112}}
{\textit{Акимова Г. Н., Вяткина С. В., Казаков В. П. }Синтаксис современного русского языка: учебник для высших заведений/ Под ред. \textit{С. В. Вяткиной.} СПб, 2013. См.: С.175—204}
{\textit{Павлова В. В.}Сравнительные конструкции в структуре простого предложения: автореф. дис. … канд. фил. наук. М., 1994.}
{\textit{Лыткина Г. В.}Сравнительные конструкции русского языка: логико-лингвистический аспект // Филологические науки. Вопросы теории и практики. № 6. 2016. См.: С.217}
{\textit{Лыткина Г. В.}Указ. соч. См.: С.218}
{Энциклопедический словарь-справочник. Выразительные средства русского языка и речевые ошибки и недочеты/ Под ред. \textit{А. П. Сковородникова.} М., 2005. С. 306}
{Энциклопедический словарь-справочник/ Под ред. \textit{А. П. Сковородникова.} М., 2005. С. 323}
{\textit{Акимова Г. Н.} Новое в синтаксисе русского языка. М., 1990. С. 100}
{\textit{Томашевский Б.В.} Стилистика. Л., 1983. См.: С. 216}
{\textit{Гореликова М.И., Магомедова Д.М.} Лингвистический анализ художественного текста. М., 1983. С. 22}
{\textit{{ Лесскис Г. А. }}{О размерах предложений в научной и художественной прозе 60-х гг. XIX века// Вопросы языкознания. М., 1962 (2). С.78}}
{\textit{{ Лесскис Г. А. }}{О размерах предложений в научной и художественной прозе 60-х гг. XIX века//Вопросы языкознания. М., 1962 (2). С.81}}
{Энциклопедический словарь-справочник. Выразительные средства русского языка и речевые ошибки и недочеты/ Под ред.\textit{А. П. Сковородникова}{. М., 2005. С.217}}
{\textit{{ Иванчикова Е. А. }}{Русский язык и советское общество: Морфология и синтаксис современного русского литературного языка. М., 1977.}\textit{{
      }}{С.280}}
{\textit{{ Валгина Н. С. }}{Актуальные проблемы русской пунктуации. М., 2004. С.106}}
{\textit{{Иванчикова Е. А. }}{Указ. соч. См.: С.287—296}}
{\textit{ Иванчикова Е. А. }{Указ. соч. См.: С.290—289}}
{\textit{{ Сковородников А. П.}}{О соотношении понятий «парцелляция» и «присоединение» / Вопросы языкознания. М., 1978 (1). С.125}}
{\textit{Сковородников А. П.}{Указ. соч. С.126}}
{\textit{{ Сковородников А. П. }}{Указ. соч. С.127}}
{\textit{{ Сковородников А. П. }}{Указ. соч. С.127}}
{{ }\textit{{Валимова Г. А. }}{Сложное предложение и сочетание предложений/ Вопросы синтаксиса современного русского литературного языка». Ростов н/Д, 1973. С.93}}
{{ }\textit{{Покусаенко В. К. }}{К вопросу о парцелляции сложных бессоюзных предложений/ Вопросы синтаксиса современного русского литературного языка. Ростов н/Д, 1973. См.: С.143—144}}
{\textit{ Золотова Г. А.} Синтаксический словарь. М., 2001. С.82}
{\textit{Золотова Г. А.} Синтаксический словарь. М., 2001. С.39}
{\textit{Земская Е. Н. Китайгородская М. В. }\textit{Ширяев Е. Н. }Русская разговорная речь. Общие вопросы. Словообразование. Синтаксис/Конситуативные высказывания// М., 1981. С.195}
{См.: Пример 2 в «Приложении» к данной работе.}
{ Вопросы истории, 1995, № 1. Политический архив XX в. Фрагменты стенограммы декабрьского пленума ЦК ВКП(б) 1936 года.\par Ссылка на электронный ресурс: http://old.memo.ru/history/1937/dec\_1936/index.htm}
{ Артизов А. Н. Сигачев Ю. В. Россия. XX век// Опубликовано в альманахе по: Никита Хрущев. 1964: Стенограммы пленума ЦК КПСС и другие документы. М.: МФД: Материк, 2007. С. 52–61.}