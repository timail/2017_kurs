\textit{Адмони В.Г.} Размер предложения и словосочетания как явлениесинтаксического строя // Вопросы языкознания. М., 1966 (4). С.112
\textit{Акимова Г. Н.,Вяткина С. В., Казаков В. П. }Синтаксис современного русского языка.Спб, 2013. См.: С.175—204
\textit{Акимова Г. Н.}Размер предложения как фактор стилистики играмматики // Вопросы языкознания. М., 1973 (2). С.79
\textit{Барт Р.} Избранные работы: Семиотика. Поэтика. М., 1994.С.387
\textit{Валгина Н. С.}Авторская модальность. Образ автора // Теориятекста. М., 2004. С.96
\textit{Валгина Н. С.}Актуальные проблемы русской пунктуации. М., 2004.С.106
\textit{Валгина Н.С. }Авторская модальность. Образ автора // Теориятекста. М., 2004. С.96—97
\textit{Виноградов В. В.}«Стиль «Пиковой дамы» // Временник Пушкинскойкомиссии, т. 2, 1936. С.75—86.
\textit{Гальперин И. Р.}Текст как объект лингвистического исследования. М,2006. С.117
\textit{Гончарова Е. А., Шишкина И. П. }Интерпретация текста. Немецкий язык. М.,Высшая школа, 2005. 368 с. 
\textit{Заика В. И.}Повествователь как компонент художественной модели // Говорящий и слушающий: Языковая личность, текст, проблемы обучения.СПб, 2001. C.384
\textit{Звягина М. Ю.}Жанровые трансформации в русской прозе второйполовины XX~--- нач. XXI в. Астрахань, 2016. С.28
\textit{Земская Е. Н. Китайгородская М. В., Ширяев Е. Н. }Русская разговорная речь. Общие вопросы. Словообразование. Синтаксис/Конситуативные высказывания // М., 1981.С.195
\textit{Золотова Г. А.}О традициях и тенденциях в современнойграмматической науке // Синтаксис современного русского языка.Хрестоматия с заданиями. СПб, 2013. С.31
\textit{Золотова Г. А.}Синтаксический словарь. М., 2001.С.82
\textit{Иванин М. И. }О стенографии или искусстве скорописи и применении её к русскому языку. Спб., 1858
\textit{Иванчикова Е. А.}}{Русский язык и советское общество: Морфология исинтаксис современного русского литературного языка. М.,1977.}{С.280}
\textit{Караулов  Ю.Н. }}{Русский язык и языковая личность. М., 1987. С.38}} 
\textit{Лесскис Г. А.}О размерах предложений в научной и художественнойпрозе 60-х гг. XIX века // Вопросы языкознания. М., 1962 (2). С.78
\textit{Лыткина Г. В.}Сравнительные конструкции русского языка: логико-лингвистическийаспект // Филологические науки. Вопросы теории и практики. Тамбов, 2016(6). См.: С.217
\textit{Мартьянова И. А.}Функционирование синтаксических единиц всовременной русской прозе // Вестник Череповецкого государственногоуниверситета. 2012. № 2 (37). С.82
\textit{Нечаева И. В., Захаренко Е. Н., Комарова Л. Н. }Новый словарь иностранных слов. М., 2008. Ссылка на электронный ресурс:https://slovar.cc/rus/inostr-nov/1431100.html
\textit{Павлова В. В.}Сравнительные конструкции в структуре простого предложения. М., 1994
\textit{Руднев В. П.}Словарь культуры XX века. М.,1997. С.238—240 
\textit{Сковородников А. П.}О соотношении понятий «парцелляция» и «присоединение» / Вопросы языкознания. М., 1978 (1). С.125
\textit{Шведова Н. Ю., Лопатина В.В. }Русская грамматика. М., 1990. С.504


%sources
%section{Источники
Вопросы истории, 1995, № 1. Политический архив XX в. Фрагменты стенограммы декабрьского пленума ЦК ВКП(б) 1936 года.\par Ссылка на электронный ресурс: http://old.memo.ru/history/1937/dec\_1936/index.htm
Артизов А. Н. Сигачев Ю. В. Россия. XX век // Опубликовано в альманахе по: Никита Хрущев. 1964: Стенограммы пленума ЦК КПСС и другие документы. М.: МФД: Материк, 2007. С. 52–61.

Российский энциклопедический словарь/ под ред. А. М. Прохорова. М., 2001 
Большая советская энциклопедия/ cл.cтатья: Соколов Н. Н., Скородумова Н. П. М., 3-е изд. 1970~--- 1978
Энциклопедический словарь-справочник. Выразительные средства русского языка и речевыеошибки и недочеты/ Под ред. А. П. Сковородникова