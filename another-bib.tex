\textit{Звягина М. Ю.}{Жанровые трансформации в русской прозе второйполовины XX~--- нач. XXI в. Астрахань, 2016. С.28}
Российский энциклопедический словарь/ под ред. А. М. Прохорова. М., 2001 
Большая советская энциклопедия/ cл.cтатья: Соколов Н. Н., Скородумова Н. П. М., 3-е изд. 1970~--- 1978
\textit{{Нечаева И. В., ЗахаренкоЕ. Н., Комарова Л. Н. }}{Новый словарь иностранныхслов. М., 2008. Ссылка на электронный ресурс:https://slovar.cc/rus/inostr-nov/1431100.html}
\textit{{Мартьянова И. А.}}{Функционирование синтаксических единиц всовременной русской прозе// Вестник Череповецкого государственногоуниверситета. 2012. № 2 (37). С.82}
\textit{{ Руднев В. П.}}{Словарь культуры XX века. М.,1997. С.238—240 }

\textit{Иванин М. И. }О стенографии или искусстве скорописи и применении её к русскому языку. Спб., 1858
 Цит. по:
\textit{{Виноградов В. В.}}{«Стиль «Пиковой дамы» // Временник Пушкинскойкомиссии, т. 2, 1936. С.75—86.}
Термин «языковаяличность» используется в значении: «Личность, выраженная в языке(текстах) и через язык; личность, реконструированная в основных своихчертах на базе языковых средств». 
\textit{{Падучева Е. В.}}{Семантические исследования. Семантика времени ивида в русском языке. Семантика нарратива.~--- М.: Языки славянскойкультуры, 1996. Изд. 2-е, 2010. С.201—202[далее в сносках: Указ. соч.1.]}
\textit{{Заика В. И.}}{Повествователь как компонент художественной модели//Говорящий и слушающий: Языковая личность, текст, проблемы обучения.СПб, 2001. C.384}
\textit{{БартР.}}{ Избранные работы: Семиотика. Поэтика. М., 1994.С.387}
\textit{{Гончарова Е. А., Шишкина И. П. }}{Интерпретация текста. Немецкий язык. М.,Высшая школа, 2005. 368 с. }
\textit{{Валгина Н.С. }}{Авторская модальность. Образ автора // Теориятекста. М., 2004. С.96—97}
\textit{{Гальперин И. Р.}}{Текст как объект лингвистического исследования. М,2006. С.117}
\textit{{Валгина Н. С.}}{Авторская модальность. Образ автора // Теориятекста. М., 2004. С.96}
\textit{{Золотова Г. А.}}{О традициях и тенденциях в современнойграмматической науке//Синтаксис современного русского языка.Хрестоматия с заданиями. СПб, 2013. С.31}
\textit{{Падучева Е. В. }}{В. В. Виноградов и наука о языкехудожественной прозы // Известия РАН. Серия литературы и языка. Т. 54,№ 3, 1995. С.43 [далее в сносках: Указ. соч. 2]}
\textit{{ ПадучеваЕ. В. }}{Указ. соч. 1. С.200—201}
\textit{{ Падучева Е. В.}}{Указ. соч. 1. С.201}
\textit{{Падучева Е. В. }}{Указ. соч. 2. См.: С.44}
\textit{{ Падучева Е. В.}}{Указ. соч. 1. См.: С.298—299}
\textit{{ Падучева Е. В.}}{Указ. соч. 1. С.297}
\textit{{ Падучева Е. В.}}{Указ. соч. 1. С.299}
\textit{{ Падучева Е. В.}}{Указ. соч. 1. С.262—265}
\textit{{ Падучева Е. В.}}{Указ. соч. 1. С.203—204}
\textit{{ Шведова Н. Ю. ЛопатинаВ. В. }}{Русская грамматика. М., 1990. С.499}
\textit{{ Падучева Е. В.}}{Указ. соч. 1. См.: С.299—300}
\textit{{ Падучева Е. В.}}{Указ. соч. 1. См.: С.300}
\textit{{ Шведова Н. Ю. Лопатина В.В. }}{Русская грамматика. М., 1990. С.504}
\textit{{ Акимова Г. Н.}}{Размер предложения как фактор стилистики играмматики// Вопросы языкознания. М., 1973 (2). С.79}
\textit{{ Адмони В.Г.}}{ Размер предложения и словосочетания как явлениесинтаксического строя// Вопросы языкознания. М., 1966 (4). С.112}
\textit{Акимова Г. Н.,Вяткина С. В., Казаков В. П. }Синтаксис современного русского языка.Спб, 2013. См.: С.175—204
\textit{Павлова В. В.}Сравнительные конструкции в структуре простого предложения. М., 1994
\textit{Лыткина Г. В.}Сравнительные конструкции русского языка: логико-лингвистическийаспект// Филологические науки. Вопросы теории и практики. Тамбов, 2016(6). См.: С.217
\textit{Лыткина Г. В. }Указ. соч.См.: С.218
\textit{{ Лесскис Г. А.}}{О размерах предложений в научной и художественнойпрозе 60-х гг. XIX века// Вопросы языкознания. М., 1962 (2). С.78}
\textit{{ Лесскис Г. А.}}{О размерах предложений в научной и художественнойпрозе 60-х гг. XIX века//Вопросы языкознания. М., 1962 (2). С.81}
 Энциклопедическийсловарь-справочник. Выразительные средства русского языка и речевыеошибки и недочеты/ Под ред. 
\textit{{ Иванчикова Е. А.}}{Русский язык и советское общество: Морфология исинтаксис современного русского литературного языка. М.,1977.}\textit{{}}{С.280}
\textit{{ Валгина Н. С.}}{Актуальные проблемы русской пунктуации. М., 2004.С.106}
\textit{{Иванчикова Е. А.}}{Указ. сочин. См.: С.287—296}
\textit{ Иванчикова Е. А.}{Указ. сочин. См.: С.290—289}
\textit{{ Сковородников А. П.}}{О соотношении понятий «парцелляция» и «присоединение» / Вопросы языкознания. М., 1978 (1). С.125}
\textit{Сковородников А. П.}{Указ. соч. С.126}
\textit{{Сковородников А. П. }}{Указ. соч.С.127}
\textit{{Сковородников А. П. }}{Указ. соч. С.127}


\textit{{ Золотова Г. А.}}{Синтаксический словарь. М., 2001.С.82}

\textit{Земская Е. Н. Китайгородская М. В.}\textit{Ширяев Е. Н. }Русская разговорная речь. Общие вопросы.Словообразование. Синтаксис/Конситуативные высказывания// М., 1981.С.195
См.: Пример 2 в«Приложении» к данной работе.

    Вопросы истории, 1995, № 1. Политический архив XX в. Фрагменты
    стенограммы декабрьского пленума ЦК ВКП(б) 1936 года.\par Ссылка на
    электронный ресурс:
    http://old.memo.ru/history/1937/dec\_1936/index.htm
 Артизов А. Н. Сигачев Ю. В. Россия. XX
    век// Опубликовано в альманахе по: Никита Хрущев. 1964: Стенограммы
    пленума ЦК КПСС и другие документы. М.: МФД: Материк, 2007. С. 52–61.