  \chapteroid{Список использованной литературы}

    \section*{Источники}
      \begin{enumerate}
        \item Артизов А. Н. Сигачев Ю. В. Россия. XX век // Опубликовано в альманахе по: Никита Хрущев. 1964: Стенограммы пленума ЦК КПСС и другие документы. М.: МФД: Материк, 2007. С. 52–61.
        \item Вопросы истории, 1995, № 1. Политический архив XX в. Фрагменты стенограммы декабрьского пленума ЦК ВКП(б) 1936 года.\par Ссылка на электронный ресурс: http://old.memo.ru/history/1937/dec\_1936/index.htm
        \item Бочоришвили Е. Только ждать и смотреть. М., 2015
      \end{enumerate}
      
    \section*{Научная литература}
      \begin{enumerate}
        \item \textit{Адмони В.Г.} Размер предложения и словосочетания как явлениесинтаксического строя // Вопросы языкознания.  1966. №4. 
        \item \textit{Акимова Г. Н., Вяткина С. В., Казаков В. П. }Синтаксис современного русского языка.СПб, 2013. 
        \item \textit{Акимова Г. Н.}Размер предложения как фактор стилистики играмматики // Вопросы языкознания. М., 1973. №2.
        \item \textit{Барт Р.} Избранные работы: Семиотика. Поэтика. М., 1994.
        \item Большая советская энциклопедия / cл.cтатья: Соколов Н. Н., Скородумова Н. П. М., 3-е изд. 1970~--- 1978
        \item \textit{Валгина Н. С.}Авторская модальность. Образ автора // Теория текста. М., 2004. 
        \item \textit{Валгина Н. С.}Актуальные проблемы русской пунктуации. М., 2004.
        \item \textit{Виноградов В. В.}«Стиль «Пиковой дамы» // Временник Пушкинской комиссии. Т. 2. 1936. 
        \item \textit{Гальперин И. Р.}Текст как объект лингвистического исследования. М,2006. 
        \item \textit{Гончарова Е. А., Шишкина И. П. }Интерпретация текста. Немецкий язык. М.,Высшая школа, 2005.
        \item \textit{Заика В. И.}Повествователь как компонент художественной модели // Говорящий и слушающий: Языковая личность, текст, проблемы обучения.СПб, 2001.
        \item \textit{Звягина М. Ю.}Жанровые трансформации в русской прозе второйполовины XX~--- нач. XXI в. Астрахань, 2016.
        \item \textit{Земская Е. Н. Китайгородская М. В., Ширяев Е. Н. }Русская разговорная речь. Общие вопросы. Словообразование. Синтаксис / Конситуативные высказывания // М., 1981.
        \item \textit{Золотова Г. А.}О традициях и тенденциях в современнойграмматической науке // Синтаксис современного русского языка. Хрестоматия с заданиями. СПб, 2013. 
        \item \textit{Золотова Г. А.}Синтаксический словарь. М., 2001.
        \item \textit{Иванин М. И. }О стенографии или искусстве скорописи и применении её к русскому языку. СПб., 1858
        \item \textit{Иванчикова Е. А.}{Русский язык и советское общество: Морфология исинтаксис современного русского литературного языка. М.,1977.}
        \item \textit{Караулов  Ю.Н. }{Русский язык и языковая личность. М., 1987. }
        \item \textit{Лесскис Г. А.}О размерах предложений в научной и художественнойпрозе 60-х гг. XIX века // Вопросы языкознания. М., 1962. №2
        \item \textit{Лыткина Г. В.}Сравнительные конструкции русского языка: логико-лингвистическийаспект // Филологические науки. Вопросы теории и практики. Тамбов, 2016. №6.
        \item \textit{Мартьянова И. А.}Функционирование синтаксических единиц всовременной русской прозе // Вестник Череповецкого государственногоуниверситета. 2012. № 2(37)
        \item \textit{Нечаева И. В., Захаренко Е. Н., Комарова Л. Н. }Новый словарь иностранных слов. М., 2008. Ссылка на электронный ресурс:https://slovar.cc/rus/inostr-nov/1431100.html
        \item \textit{Павлова В. В.}Сравнительные конструкции в структуре простого предложения. М., 1994
        \item Российский энциклопедический словарь/ под ред. А. М. Прохорова. М., 2001 
        \item \textit{Руднев В. П.}Словарь культуры XX века. М.,1997.
        \item \textit{Сковородников А. П.}О соотношении понятий «парцелляция» и «присоединение» / Вопросы языкознания. М., 1978. №1.
        \item \textit{Шведова Н. Ю., Лопатина В.В. }Русская грамматика. М., 1990. 
        \item Энциклопедический словарь-справочник. Выразительные средства русского языка и речевые ошибки и недочеты/ Под ред. А. П. Сковородникова
      \end{enumerate}